\documentclass{beamer}
\usetheme{CambridgeUS}
\usecolortheme{default}
\setbeamercolor{itemize item}{fg=darkred!80!black}


\makeatletter
\setbeamertemplate{footline}
{
  \leavevmode%
  \hbox{%
  \begin{beamercolorbox}[wd=.333333\paperwidth,ht=2.25ex,dp=1ex,center]{author in head/foot}%
    \usebeamerfont{author in head/foot}F. Massa, A. Fagiolini
  \end{beamercolorbox}%
  \begin{beamercolorbox}[wd=.333333\paperwidth,ht=2.25ex,dp=1ex,center]{title in head/foot}%
    \usebeamerfont{title in head/foot}Monitoring update
  \end{beamercolorbox}%
  \begin{beamercolorbox}[wd=.333333\paperwidth,ht=2.25ex,dp=1ex,right]{date in head/foot}%
    \usebeamerfont{date in head/foot} 28th of March, 2017\hspace*{2em}
    \insertframenumber{} / \ref{Conclusions}\hspace*{2ex} 
  \end{beamercolorbox}}%
  \vskip0pt%
}
\makeatother

\setbeamercolor{section in head/foot}{fg=white, bg=darkred!95!white}

\setbeamercolor{palette quaternary}{use=structure, bg=darkred!80!black} % changed this


\setbeamertemplate{headline}{%
\leavevmode%
  \hbox{%
    \begin{beamercolorbox}[wd=\paperwidth,ht=2.5ex,dp=1.125ex]{palette quaternary}%
    \insertsectionnavigationhorizontal{\paperwidth}{}{\hfill \hfill}
    \end{beamercolorbox}%
  }
}

\title{Highway simulator: \\
Monitoring update}
%\subtitle{Laurea Magistrale in Fisica}
%\author{Federico Massa}
%\institute{\large{Laurea Magistrale in Fisica \\
%Universit\`a di Pisa}}
\date{\small March, 28th 2017}

%\setbeameroption{show notes}
\setbeameroption{hide notes}
\setbeamertemplate{navigation symbols}{}

\usepackage{tikz}
\usetikzlibrary{decorations.pathreplacing}
\usetikzlibrary{positioning, calc}
\newcommand{\tikzmark}[1]{\tikz[overlay,remember picture] \node (#1) {};}

\usepackage{amsmath}% mathtools includes this so this is optional
\usepackage{mathtools}
\usepackage[export]{adjustbox}
\usepackage{multimedia}
\usepackage[utf8x]{inputenc}
\usepackage{tcolorbox}
\definecolor{dred}{RGB}{200, 0, 0}

\begin{document}

%-----------------------------------------

{\setbeamertemplate{footline}{}
\setbeamertemplate{headline}{}
\begin{frame}
\titlepage
\bigskip
\medskip
\centering\small Federico Massa, Adriano Fagiolini

\end{frame}}
\addtocounter{framenumber}{-1}

\begin{frame}
\frametitle{Why we needed an update}
\textbf{What we were doing}
\begin{itemize}
	\item We were trying to monitor the behaviour of a vehicle
		  with a discrete states automaton running different
		  controllers.
 	\item The recognition of the discrete state was made via 
 		  a forward prediction of the vehicle's continuous state,
 		  that was compared with the controller model that we
 		  supposed to know (first mayor problem)
  	\item The monitoring itself was achieved by evaluating the
  		  rules (that we also supposed to know from the constructor, second 
  		  mayor problem) that determined the transition between
  		  discrete states. Uncertain cases were due to the 
  		  partial vision of the observer with respect to the 
  		  monitored vehicle's surroundings.
\end{itemize}
\end{frame}

\begin{frame}
\frametitle{Why we needed an update}
\textbf{What we do now}
\begin{itemize}
	\item We monitor the behaviour of a vehicle based on the
		  abstract features of its movements. This way, we can
		  monitor a vehicle regardless of the dynamic model 
		  (it could also be manually-driven)
 	\item Instead of the recognition of the discrete state,
 		  we try to recognize the abstract \textbf{action} 
 		  that the vehicle is pursuing. 
  	\item The monitoring is now based on the definition
  		  of some basic \textbf{social rules}, that 
  		  are defined in an abstract way as the list of
  		  conditions that make a certain action forbidden.
\end{itemize}

To achieve this, we only need:
\begin{itemize}
\item The sensor data of all the vehicles visible to the observer (x, y 
      on the highway plane, at least) + some kind of vehicle ID to 
      follow its trajectory.
\item A list of possible actions that the monitored vehicle can do.
\item The implementation of the social rules that are being monitored.
\end{itemize}
\end{frame}

\begin{frame}
\frametitle{First step: the action recognition}
An action is abstractly defined as a vehicle behaviour. Examples of
it include 
\begin{itemize}
\item \textit{left lane change}
\item \textit{overtake}
\item \textit{brake}
\item ...
\end{itemize}

We could also imagine a more sophisticated action recognition system
to include nuances of these basic actions, like \textit{slow left lane change} or
\textit{abrupt left lane change}.\\

\textbf{What the observer need to know:}

The observer vehicle is supposed to have a list of all the

\end{frame}


\begin{frame}\label{Conclusions}
\end{frame}
\end{document}

