\documentclass{beamer}
\usetheme{CambridgeUS}
\usecolortheme{default}
\setbeamercolor{itemize item}{fg=darkred!80!black}

\makeatletter
\setbeamertemplate{footline}
{
  \leavevmode%
  \hbox{%
  \begin{beamercolorbox}[wd=.333333\paperwidth,ht=2.25ex,dp=1ex,center]{author in head/foot}%
    \usebeamerfont{author in head/foot}F. Massa
  \end{beamercolorbox}%
  \begin{beamercolorbox}[wd=.333333\paperwidth,ht=2.25ex,dp=1ex,center]{title in head/foot}%
    \usebeamerfont{title in head/foot}Monitoring update
  \end{beamercolorbox}%
  \begin{beamercolorbox}[wd=.333333\paperwidth,ht=2.25ex,dp=1ex,right]{date in head/foot}%
    \usebeamerfont{date in head/foot} May, 4th 2017\hspace*{2em}
    \insertframenumber{} / \ref{Conclusions}\hspace*{2ex} 
  \end{beamercolorbox}}%
  \vskip0pt%
}
\makeatother

\setbeamercolor{section in head/foot}{fg=white, bg=darkred!95!white}

\setbeamercolor{palette quaternary}{use=structure, bg=darkred!80!black} % changed this


\setbeamertemplate{headline}{%
\leavevmode%
  \hbox{%
    \begin{beamercolorbox}[wd=\paperwidth,ht=2.5ex,dp=1.125ex]{palette quaternary}%
    \insertsectionnavigationhorizontal{\paperwidth}{}{\hfill \hfill}
    \end{beamercolorbox}%
  }
}

\title{Highway simulator: \\
Monitoring update}
%\subtitle{Laurea Magistrale in Fisica}
%\author{Federico Massa}
%\institute{\large{Laurea Magistrale in Fisica \\
%Universit\`a di Pisa}}
\date{\small May, 4th 2017}

%\setbeameroption{show notes}
\setbeameroption{hide notes}
\setbeamertemplate{navigation symbols}{}

\usepackage{tikz}
\usetikzlibrary{decorations.pathreplacing}
\usetikzlibrary{positioning, calc}
\newcommand{\tikzmark}[1]{\tikz[overlay,remember picture] \node (#1) {};}

\usepackage{amsmath}% mathtools includes this so this is optional
\usepackage{mathtools}
\usepackage[export]{adjustbox}
\usepackage{multimedia}
\usepackage{multirow}
\usepackage[utf8x]{inputenc}
\usepackage{enumerate}
\usepackage{tcolorbox}
\definecolor{dred}{RGB}{200, 0, 0}

\begin{document}

%-----------------------------------------

{\setbeamertemplate{footline}{}
\setbeamertemplate{headline}{}
\begin{frame}
\titlepage
\bigskip
\medskip
\centering\small Federico Massa, Adriano Fagiolini, Antonio Bicchi, Lucia Pallottino
\end{frame}}
\addtocounter{framenumber}{-1}

\section{Introduction}

\begin{frame}
\frametitle{Objectives}

\textbf{Build a software that runs on a vehicle ({\color{dred}Monitor}):}

\begin{itemize}
\item[-] that is capable to detect high-level behaviours of the neighboring vehicles based on the study of their trajectory;
\item[-] without requiring any non-trivial knowledge on the observed vehicles' dynamic model;
\item[-] able to evaluate the compliance of this behaviour with a set of
\textbf{social rules}
\end{itemize}

How?\\

\begin{itemize}
	\item We monitor the behaviour of a vehicle based on the
		  abstract features of its movements (\emph{actions})
  	\item The \emph{social rules} 
  		  are defined in an abstract way as the list of
  		  conditions that make a certain action forbidden.
\end{itemize}
\end{frame}

\begin{frame}
\frametitle{Monitor requirements}

To achieve this, we only need:
\begin{itemize}
\item The sensor data of all the vehicles visible by the observer (x, y 
      on the highway plane, at least);
\item Some kind of vehicle ID to 
      follow its trajectory;
\item A list of possible actions that the monitored vehicle can do;
\item A specification of the social rules that are being monitored.
\end{itemize}

\end{frame}

\begin{frame}
\frametitle{Simulator}

To study this algorithm, we use a C++ simulator that instantiates several vehicles in a highway-like environment. 
In the configuration file it is possible to choose, for every vehicle:
\begin{columns}
\begin{column}{0.6\textwidth}
\begin{itemize}
\item Initial continuous and discrete state;
\item Geometrical parameters \\(Type of vehicle);
\item Dynamic model (Physical Layer type);
\item Automaton type;
\item Magnitude of sensor errors.
\end{itemize}
\end{column}
\begin{column}{0.4\textwidth}
\centering
\includegraphics[scale=0.15]{vehicletypes}
\end{column}
\end{columns}


\begin{columns}[c]
\begin{column}{0.5\textwidth}
\centering
\includegraphics[width=.6\linewidth]{coordinates}
\end{column}
\begin{column}{0.5\textwidth}
\flushleft
\includegraphics[width=.4\linewidth]{Automaton}
\end{column}
\end{columns}

\end{frame}

\section{Action recognition}


\begin{frame}
\frametitle{First step: the action recognition}
An action is abstractly defined as a vehicle behaviour in a \textbf{finite interval of time}. Examples of
it include 
\begin{columns}
\begin{column}{0.4\textwidth}
\begin{itemize}
\item \textit{Travel}
\item \textit{Left/Right lane change}
\item \textit{Left/Right overtake}
\item \textit{Brake}
\end{itemize}
\end{column}
\begin{column}{0.6\textwidth}
\flushright
\includegraphics[scale=0.4]{actionsTimeline}
\end{column}
\end{columns}

\bigskip

We can also imagine a more sophisticated action recognition system
to include more complicated actions, like \textit{slow left lane change} or
\textit{abrupt left lane change}.\\

\begin{block}{\textbf{What the monitor needs to know:}}
The monitor must have a list of possible actions that the 
observed vehicle can do. The action recognition system is now \textbf{independent} from the monitored vehicle's dynamic model. 
\end{block}

\end{frame}

\begin{frame}
\frametitle{How the action recognition works}
In order to recognize an action, the monitor needs to:
\begin{itemize}
\item Identify the observed vehicle;
\item Keep track of the visible vehicles' sensor data for a finite interval of time.
\end{itemize}

Each action is recognized independently $\rightarrow$ concurrent actions are possible.

\bigskip

An \textbf{Action Manager} ($\mathbf{\Phi}$) continuously listens to each action ($a_1$, $a_2$, ...) and verifies if the trajectory of the monitored vehicle ($q_1, ..., q_N$) is compatible with that action. If it is, the action is \emph{triggered}.

\end{frame}

\begin{frame}
\frametitle{The Action Manager structure}
The ActionManager initializes several \textit{listeners}, i.e. actions that the system can recognize.
\bigskip
\bigskip
\bigskip

\centering
\includegraphics[scale=0.5]{ActionManager}

\end{frame}

\begin{frame}
\frametitle{listen() cycle}
If an action is registered in the ActionManager listener list, it enters a cycle:\\

\centering
\includegraphics<1>[scale=0.6]{ListeningCycle4}
\includegraphics<2>[scale=0.4]{ListeningSignal}


\end{frame}

\begin{frame}
\frametitle{LeftAction example}

We use a N-point trajectory ($q_1, ..., q_N$) and the following parameters:
\begin{description}
\item[\color{dred}$\Delta y$:] Transversal distance covered in the recorded trajectory $(y_N - y_1)/N$;
\item[\color{dred}$\sigma_y$:] Standard deviation of transversal position.
\item[\color{dred}f:] constant between 0 and 1 (needs tuning);
\item[\color{dred}$\Delta y_{tolerance}$:] transversal position tolerance on central position (needs tuning);
\item[\color{dred}$\epsilon_y$:] transverse position sensor error. 
\end{description}

\bigskip

\begin{description}
\item[triggerCondition()] 
$ \Delta y > f\cdot v_{max}\cdot \Delta t_{sim. step}$ \\
 $\ \ \ \ \ \ \ |y_{1} - y_{initLane}| < \Delta y_{tolerance} + 3\cdot\epsilon_y$

\item[endCondition()] 
$\ \ \ \ |\mu_y - y_{targetLane}| < \Delta y_{tolerance} + 3\cdot\epsilon_y/\sqrt{N}$ \\
$\ \ \ \ \ \ \ \ \sigma_y < \epsilon_y$

\item[abortCondition()] \ \ After trigger time, the vehicle has disappeared.
 
\end{description}
\end{frame}

\begin{frame}
\frametitle{LeftAction video}

\movie[externalviewer]{\includegraphics[scale=0.5]{leftActionVideo2.png}}{leftActionVideo2.avi}

\end{frame}

\begin{frame}
\frametitle{Complete action example}

\movie[externalviewer]{\includegraphics<+>[scale=0.5]{ActionExample/0.png}}{ActionExample/actionVideo2.avi}
\includegraphics<+>[scale=0.5]{ActionExample/7.png}
\includegraphics<+>[scale=0.5]{ActionExample/28.png}
\includegraphics<+>[scale=0.5]{ActionExample/40.png}
\includegraphics<+>[scale=0.5]{ActionExample/90.png}
\includegraphics<+>[scale=0.5]{ActionExample/125.png}
\includegraphics<+>[scale=0.5]{ActionExample/150.png}


\bigskip
\bigskip
\bigskip
\bigskip	

\includegraphics[scale=0.5]{Results2}
\end{frame}

\section{Social rules monitoring}
\begin{frame}
\frametitle{Second step: implementation of social rules}
\begin{description}
\item[\textbf{\color{black}Observation:}] there are not unique correct behaviours, but there are {\color{dred}wrong} ones.
\end{description}

\onslide<2->
{
In practice, a set of social rules is composed by single rules with the
following features:
\begin{description}
\item[Category] A string useful to define groups of rules, such as "Safety rules", "Lane change rules", ...}
\item<3->[Event list] The rule is activated when at least one event is true.  Each event is composed by subevents and is activated when all subevents are true. Each subevent checks a specific logical condition on the monitored vehicle and its neighbors and can have a specific \textbf{area} of influence. \\
\item<4->[Eval mode] Specifies when the rule must be evaluated (when the action is triggered or continuously).
\end{description}
\end{frame}

\begin{frame}
\frametitle{Example: Italian-like highway rules}
\textbf{Safety rules}\\
\begin{itemize}
\item Safety distance, 1 event: ``Someone is in front of the observed vehicle at less than x meters'';
\item Maximum speed, 1 event: ``The observed vehicle is faster than x km/h''
\end{itemize}

\textbf{Left lane change rules}
\begin{itemize}
\item 1 event: ``Nobody is in front of the observed vehicle'';
\item 1 event: ``Left lane is occupied'';
\item 1 event: ``The observed vehicle is on the maximum lane''.
\end{itemize}

\textbf{Right lane change rules}
\begin{itemize}
\item 1 event: ``Right lane is occupied'';
\item 1 event: ``The observed vehicle is on the minimum lane''.
\end{itemize}

\textbf{Travel rules}
\begin{itemize}
\item 1 event: ``Right lane is free'';
\end{itemize}

\textbf{Left/Right overtake} allowed.
\end{frame}

\begin{frame}
\frametitle{Action Manager and Rule Monitor}
An object called \textbf{Rule Monitor} is used to check if the rules are 
violated, as the Action Manager ``listens'' to each action.\\

These two objects are independent, but they can communicate via the \textbf{rules categories}.\\

$\rightarrow$ Each action has a set of rule categories that it must follow. 

{Ex. LeftAction must follow the ``Safety'' rules and the ``Left Lane Change'' rules.}

\onslide<2->{
\begin{alertblock}{}This has several pros:\\
\begin{itemize}
\item Avoid the difficult abstraction task that links the specific action to the abstract rule it should follow;
\item Keeps the action and the rule systems well separated;
\item Different actions can share some rule categories.
\end{itemize}
}
\end{alertblock}

\onslide<3->{
$\rightarrow$ This also allows to make complex action managers without changing the rule system.}


\end{frame}

\begin{frame}
\frametitle{Action and rules}

\centering
\begin{tabular}{|c|c|} \hline
\textbf{Action} & \textbf{Rule categories} \\ \hline 
\multirow{2}{*}{LeftAction} & Safety \\ 
	& Left Lane Change \\ \hline
\multirow{2}{*}{RightAction} & Safety \\ 
	& Right Lane Change \\ \hline 
\multirow{2}{*}{TravelAction} & Safety \\ 
	& Cruise \\ \hline 
\multirow{2}{*}{LeftOvertakeAction} & Safety \\ 
	& Cruise \\ \hline 
\multirow{2}{*}{RightOvertakeAction} & Safety \\ 
	& Cruise \\ \hline 
\end{tabular}
\end{frame}

\begin{frame}

\frametitle{The Rule Monitor structure}
Just alike the Action Manager, the Rule Monitor object is used to initialize
the rule system and verify that the right rules are checked. At each instant:\\

\setbeamertemplate{enumerate items}[default]
\begin{enumerate}
\item<+-> Ask the Action Manager the list of triggered actions;\\ 
\item<+-> Extract the list of rule categories to be checked by merging the
	list of rule categories of each action;
\item<+-> For each rule category, take each rule and consider its ``Evaluation Mode'';
\item<+-> If the Evaluation Mode is on TRIGGER, the rule is checked only at trigger time. If it is on CONTINUOUS, the rule is checked every time until the action ends (e.g. Safety rules are generally in CONTINUOUS mode);
\item<+-> A rule is \textbf{True} if a wrong behaviour was spotted (at least one of the events is true);\\
\textbf{Uncertain} if the observer cannot tell 
if a rule is verified or not due to hidden areas (no events are true, at least one uncertain);\\
\textbf{False} if the behaviour is correct  (all events are false).
\end{enumerate}
\end{frame}

\begin{frame}
\frametitle{LeftAction example}

***Specific left action rules are evaluated at trigger time. For graphical simplicity only the areas are, instead, shown continuously.

Without sensor errors:\\
\movie[externalviewer]{\includegraphics[scale=0.5]{leftActionVideo2.png}}{monitorLeftActionVideoNoErrors2.avi}

\bigskip
\bigskip

With sensor errors:\\
\movie[externalviewer]{\includegraphics[scale=0.5]{leftActionVideo2.png}}{monitorLeftActionVideoErrors2.avi}

\end{frame}

\section{Conclusions}
\begin{frame}
\frametitle{Remarks on the C++ code}
The C++ code is mostly written using polymorphism, so that it is
very easy to make a more complex system without having to understand the
underlying code:\\

\begin{itemize}
\item Specific actions inherit from the Action class. To add a new action,
	it is only necessary to manually write the trigger, end and abort 			    conditions.
\item Specific set of rules inherit from the SocialRules class. After   		    manually
	writing a new set of rules it is possible to change the monitored rules
	by changing a single line of code.
\end{itemize}

A similar concept applies when wanting to change the vehicle geometry, dynamics or automaton. The code is written keeping flexibility in high regard.


\end{frame}

\begin{frame}\label{Conclusions}
\frametitle{Summing up}
The simulator should now be able to:
\begin{itemize}
\item Recognize and record list of actions done by the monitored vehicles;
\item Verify, if possible, if the monitored vehicle is following the rules;
\item Do this no matter the vehicle size, dynamic model, state transition rules (could also be manually driven in theory).
\end{itemize}


\begin{alertblock}{Critical point:}
Action parameters tuning
\end{alertblock}

Further studies:
\begin{itemize}
\item Use of consensus to improve the observer dataset (both to reduce sensor errors and to reveal the content of the hidden areas);
\item Reputation system: how to deal with infractions?
\item Roomba?
\item Possible application in Roborace?
\item Besides highways? 
\end{itemize}


\end{frame}
\end{document}

